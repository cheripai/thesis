%% ----------------------------------------------------------------
%% Thesis.tex -- MAIN FILE (the one that you compile with LaTeX)
%% ---------------------------------------------------------------- 

% Set up the document
\documentclass[a4paper, 11pt, oneside]{Thesis}  % Use the "Thesis" style, based on the ECS Thesis style by Steve Gunn
\graphicspath{Figures/}  % Location of the graphics files (set up for graphics to be in PDF format)

% Include any extra LaTeX packages required
\usepackage[square, numbers, comma, sort&compress]{natbib}  % Use the "Natbib" style for the references in the Bibliography
\usepackage{verbatim}  % Needed for the "comment" environment to make LaTeX comments
\usepackage{vector}  % Allows "\bvec{}" and "\buvec{}" for "blackboard" style bold vectors in maths
\hypersetup{urlcolor=blue, colorlinks=true}  % Colours hyperlinks in blue, but this can be distracting if there are many links.

%% ----------------------------------------------------------------
\begin{document}
\frontmatter      % Begin Roman style (i, ii, iii, iv...) page numbering

%% ----------------------------------------------------------------
\Declaration{

\addtocontents{toc}{\vspace{1em}}  % Add a gap in the Contents, for aesthetics

I have read the attached thesis proposal and, in my opinion, it
proposes work which is adequate in depth and scope to serve as the
culminating experience for the Master's Degree in Computer Science.
I would agree to chair this committee or serve thereon.\\

 
Thesis\\Advisor:\\
\rule[1em]{38em}{0.5pt}

Possible\\Committee\\Member:\\
\rule[1em]{38em}{0.5pt}

Possible\\Committee\\Member:\\
\rule[1em]{38em}{0.5pt}

}
\clearpage

% The Abstract Page
\addtotoc{Abstract}  % Add the "Abstract" page entry to the Contents
\abstract{
\thispagestyle{plain}
The consistently increasing population entails a proportionately increased demand for food, or agricultural production.
However, such a demand for agricultural production cannot be easily met due to limited natural resources.
For example, at the current state of affairs, California does not have enough water to sustain all agricultural activities \cite{Medellin}.
Furthermore, these limited natural resources are needlessly diminished by careless agricultural practices.
Such practices have further ecological implications, evidenced by the excess use of fertilizers, such as nitrogen, leading to groundwater leaching \cite{Lin2001}.
In this work, we seek to address these problems by developing a system that utilizes unmanned aerial vehicles (UAVs) and machine learning to analyze crop quality.
The data acquired from the UAV is processed by our algorithm, which informs end-users of the suggested treatments.
Specifically, the algorithm determines if additional irrigation or fertilization is necessary for each plant.
By developing an automated system for crop quality analysis, we conserve physical resources while reducing human labor and the ecological impact of agriculture.
%
% As our population increases, agricultural production must increase accordingly.
% Despite this growth in population, we are still limited by our natural resources. 
% Furthermore, careless agricultural practices can damage these natural resources.
% At the current state of affairs, California does not have enough water to sustain all agricultural activities and %SOURCE!!!! 
% excessive use of fertilizers, such as nitrogen, can lead to groundwater leaching %SOURCE!!!!.
% In this work, we seek to address these problems by developing a system that utilizes unmanned aerial vehicles (UAVs) and machine learning to analyze crop quality.
% The data from the UAV is processed by our algorithm which informs end-users of the suggested treatments.
% Specifically, the algorithm determines if additional irrigation or fertilization is necessary for each plant.
% By developing an automated system for crop quality analysis, we conserve physical resources while reducing human labor and the ecological impact of agriculture.
}

\clearpage  % Abstract ended, start a new page
%% ----------------------------------------------------------------
\tableofcontents  % Write out the Table of Contents

%% ----------------------------------------------------------------
\setstretch{1.5}  % Set the line spacing to 1.5, this makes the following tables easier to read
\clearpage  % Start a new page

\mainmatter	  % Begin normal, numeric (1,2,3...) page numbering
\pagestyle{fancy}  % Return the page headers back to the "fancy" style

% Include the chapters of the thesis, as separate files
% Just uncomment the lines as you write the chapters

\chapter{Introduction}
With rapid advances in the field of machine learning, we are discovering a myriad of innovative applications of the technology. In particular, agriculture is a field that machine learning is beginning to revolutionize. Our motivations for improving agriculture are clear: feeding more people, utilizing fewer resources, and reducing the need for physical labor.

This work aims to utilize data collected from sensors to analyze the health of various crops. Particularly, we are interested in utilizing sensors mounted on unmanned aerial vehicles (UAVs) to collect data. This is due to the speed at which these systems can collect data and the accessibility for our end users. Given this data, we leverage convolutional neural networks (CNNs) to make predictions about the crops to determine whether there is an adequate amount of water and fertilizer. This system will enable farmers to target the specific plants that require water or fertilizer treatments, thus conserving resources. Applications of our technology will also improve overall crop health, which reduces unnecessary plant waste.

Our work focuses on two crops specifically: lettuce and oranges. California is the largest producer of lettuce in the U.S. with 73\% of total lettuce acreage \cite{Lettuce}. Moreover, over 30\% of U.S. orange production comes from California \cite{Orange}. Since these crops are clearly significant to the California agriculture industry, our work will have the largest impact as such. % Introduction

\chapter{Literature Survey}


\section{Precision Agriculture}
Precision agriculture (PA) is an information-based and production-based farming system that has the goal of increasing farm production efficiency, productivity and profitability. Use of PA can lead to the reduction of the negative impacts of farming such as excessive water and chemical usage \cite{Liaghat2010}. Developments in PA are critical in order to develop a more sustainable approach to agriculture.


\subsection{Remote Sensing}
Remote Sensing (RS) refers to acquisition and interpretation of data that was obtained by sensors that are not in physical contact with the object being observed. These sensors include typically include standard cameras as well as hyperspectral and multispectral cameras \cite{Liaghat2010}. Hyperspectral and multispectral cameras both capture light beyond the visible light range, but they differ in that hyperspectral cameras have higher spectral resolution. This means that data is collected from more spectral channels, typically 200 or more compared to the 3-10 channels of multispectral data \cite{Abuleil}. Hyperspectral and multispectral data is useful in that it can be utilized to determine crop cover, crop health, and soil moisture. A study in Zhonglianchua, China showed that multispectral imagery was capable of estimating crop yields with $r^2 = 0.86$ between the predicted and actual yield \cite{Pan2009}.


\subsection{Normalized Difference Vegetative Index}
% Make subsection indices add ENDVI%%%%%%%%%%%%%%%
% ENDVI = ((NIR+Green) - (2*Blue)) / ((NIR+Green) + (2*Blue))
The normalized difference vegetative index (NDVI) is a measure of photosynthetic output. NDVI is based on the principle that healthy plants strongly absorb radiation in the visible region, otherwise known as the photosynthetically active radiation (PAR), of the spectrum while strongly reflecting radiation in the near infrared region (NIR) of the spectrum. As such, we can utilize hyperspectral or multispectral sensors to record reflectance in the corresponding ranges to determine the health of plants. The index can be calculated as:
\begin{equation} NDVI = \frac{NIR - PAR}{NIR + PAR} \end{equation}
Practically, only the red range of the spectrum is utilized rather than all of the PAR, making the index: 
\begin{equation} NDVI = \frac{NIR - Red}{NIR + Red} \end{equation}
The index will yield values in the range of $-1$ to $1$, where $1$ indicates strongest vegetative growth \cite{Ryan}. 

\begin{figure}
    \centering
    \includegraphics[width=1.0\textwidth]{images/ndvi.png}
    \caption{Left: RGB image of vegetation. Right: NDVI of the same vegetation, where hotter colors correspond to higher NDVI \cite{Abuleil}.}
    \label{ndvi_map}
\end{figure}

\subsection{Unmanned Aerial Systems}
Until recently, remote sensing has been performed by satellites, airplanes, balloons, and helicopters equipped with optical and near infrared sensors. However, there are several problems associated with these platforms. Firstly, due to the cost, these platforms are inaccessible to many farmers. Secondly, it is difficult to use these platforms on a regular basis. Ideally, farmers would collect data on their crops daily, analyze the data, and implement the necessary treatments by the next day. Finally, due to the high altitude from which the data is collected, the crops can be obscured by cloud cover or the data can suffer from low resolution. High resolution imagery is necessary for tasks such as weed detection. With weed detection, weeds must be located within an order of centimeters so that farmers can quickly find them in the fields \cite{Zhang2012}. In recent years, unmanned aerial systems have been purported as a solution to these problems.

\begin{figure}
    \centering
    \includegraphics[width=1.0\textwidth]{images/uav.png}
    \caption{Examples of UAV. \textbf{a} powered glider, \textbf{b} powered parachute, \textbf{c} helicopter, \textbf{d} fixed wing aircraft, \textbf{e} Draganflyer X8 quadrocopter, \textbf{f} Aeryon Scout quadrocopter \cite{Zhang2012}}
    \label{UAV}
\end{figure}

Unmanned aerial vehicles (UAV) are small aircraft that can be piloted remotely usually taking the form of fixed wing airplanes, helicopters, and multicopters as seen in Figure \ref{UAV}. UAV have introduced a new paradigm of remote sensing referred to as low altitude remote sensing (LARS). These systems are characterized by their low operational costs, near real-time image acqusition, and high resolution imagery. However, UAV are not without their problems. Due to the relatively low altitude of the aircraft, many images must be captured to capture all of the field. With these images, it is necessary to perform image mosaicing, which adds a level of difficulty to the use of these systems. Additional issues arise from aviation regulations. In the United States, a Certificate of Authorization must be acquired and the UAV must always be in the view of the operator \cite{Zhang2012}. Despite these challenges, it seems that UAV are the best tools at our disposal for remote sensing.


\section{Machine Learning}
Machine learning is a set of methods that can automatically detect patterns in data, and use these patterns to predict future data or perform decision making \cite{Murphy}. With recent increases in data collection and computing performance, machine learning has found its way into nearly every part of modern technology. It is used in web searches, translation, image identification, translation, and consumer electronics \cite{Nature_DL}.

Within the realm of machine learning, there exist two main types: supervised learning and unsupervised learning. 

Supervised learning, the more common of the two, involves learning a mapping from an input $x$ to an output $y$, given a set of input-output pairs $D = \{(x_i, y_i)\}^N_{i=1}$. Where $D$ is the training set and $N$ is the number of examples in the training set \cite{Murphy}. Within supervised learning, problems are typically represented in two different ways: classification and regression. A classification problem has the goal of mapping an input into a discrete category. For example, given an image of a cat or dog $x$, predict the type of animal $y \in \{cat, dog\}$. A regression problem involves mapping an input to a continuous valued output \cite{Murphy}. For instance, given the height of a person $x$, predict the person's weight $y$, where $y \in \mathbb{R}$.

Unsupervised learning only considers the data $x$ without any targets $y$. The purpose behind this is to find any structure in the data to help perform the desired task. The typical tasks in unsupervised learning are clustering and dimensionality reduction. Clustering, as the name suggests, involves finding groups within the dataset $D$. For example, in e-commerce, users are often clustered into different groups based on their purchasing behavior. These learned clusters are then used for targeted advertising. Dimensionality reduction projects high dimensional data into lower dimensional space \cite{Murphy}. A simple example would be taking 3d data and reducing it into 2d space so that it can be easily displayed in a graph.

With any machine learning problem, it is important to define an objective function. This function will measure the error of the machine learning model's predictions from the ideal results. By doing this, it allows the model to update its parameters to reduce the error of the objective function. These parameters, or weights, can be seen as knobs that define the input-output function of the model. Most machine learning models update their parameters by finding the gradient vector of the objective function with respect to the model's parameters. Once the gradient vector is found, it is negated which makes it indicate the direction to change the parameters to decrease the error from the objective function. The negated gradient vector is then element-wise summed with the parameters of the model. This process describes a common optimization method in machine learning called Gradient Descent \cite{Nature_DL}. 

\subsection{Deep Learning}

Deep learning is a form of machine learning that has had many successes in recent years. This success is validated by its dominance in image and speech recognition competitions \cite{Nature_DL}. Older machine learning methods rely on domain experts to design feature extractors that transform raw data (such pixels of an image) into a representation that is easy for the algorithm to process. On the other hand, deep learning methods perform representation learning, which means that the algorithms will automatically discover good representations for the input data. Furthermore, deep learning methods are essentially a composition of many layers of representation learning modules. Each of these modules transform the input representation into a higher, more abstract level. With the composition of these transformations, complex functions can be learned.

\subsection{Feedforward Neural Networks}

\begin{figure}
    \centering
    \includegraphics[width=0.5\textwidth]{images/mlp.png}
    \caption{A neural network that performs multiple levels of transformations \cite{Nature_DL}.}
    \label{mlp}
\end{figure}
Feedforward neural networks are the basis for various deep learning architectures. Neural networks, as they are concisely referred to, are composed of layers of units that compute a weighted sum of the inputs, which come from the previous layer, followed by a non-linear function. These layers are typically called dense layers or fully-connected layers. Currently, the most commonly used non-linear function is the rectified linear unit (ReLU), which takes the form of $f(z) = max(z, 0)$ \cite{Nature_DL}. Layers that are neither input nor output are considered to be hidden layers; these layers can take an arbitrary number nodes. An example neural network can be seen in Figure \ref{mlp}.

\subsection{Convolutional Neural Networks}
Convolutional neural networks (CNN) are inspired by the visual cortex and, as such, are optimal for processing images and other spatial data. These models did not become popular until the ImageNet competition of 2012. During this competition, a CNN was trained on millions of images of 1,000 different classes and achieved a result of half of the error of the best competing approaches. Since this success, it brought about a revolution in computer vision with CNNs being the dominant approach for recognition and detection tasks. Recently, CNNs have even approached human performance on some tasks \cite{Nature_DL}.

Convolutional neural networks are typically composed of three types of layers: convolution, pooling (sub-sampling), and dense. A convolutional layer involves sliding a fixed size filter, which can be represented as a matrix, across the input map while doing element-wise multiplication followed by a sum of the products. This operation works well with images because there is a high correlation between the local groups of values and because features in an image are invariant to location. The most common form of pooling is called max-pooling, which takes only the maximum value within a local patch of units to be in the output map. This has the effect of reducing the dimension of the representation and creating invariance to shifts and distortions \cite{Nature_DL}.

\begin{figure}
    \centering
    \includegraphics[width=0.8\textwidth]{images/conv.png}
    \caption{Convolutional layer followed by a pooling layer. \cite{Bishop2007}}
    \label{conv}
\end{figure}

Now that CNNs are the dominant method for vision related tasks, great effort is going into researching how to build the best performing model. There are many pieces of a CNN that can be adjusted. Firstly, the number of convolutional layers must be determined. Within the convolutional layers, the designer must consider the size of the filters, number of filters, and stride of the filters. Next, the number of max-pooling layers must be determined as well as the placement relative to the convolutional layers. At the end of the model, the number of dense layers as well as the number of units in each dense layer must be decided. Alternatively, some models do not even use dense layers at all. Besides configuring these existing layers, additional research is also going into new types of layers and new types of activation functions. In the following sections, we will explore several of the best CNN models.


\subsubsection{VGG}
\begin{figure}
    \centering
    \includegraphics[width=0.8\textwidth]{images/vgg_arch.png}
    \caption{Architecture of VGG (typically D or E are selected). conv3-64 refers to a 3x3 convolution with 64 filters. FC-4096 refers to a dense layer with 4096 units \cite{vgg}.}
    \label{vgg}
\end{figure}
VGG is a CNN architecture that achieved first place in the ImageNet 2014 image localization task and second place in the image classification task. It is a fairly simple model as it consists almost entirely of 3x3 convolutional filters \cite{vgg}. It seems to be the most commonly chosen architecture for computer vision related tasks. This is likely due to its simplicity and availability in deep learning frameworks.

\subsubsection{ResNet}
\begin{figure}
    \centering
    \includegraphics[width=0.65\textwidth]{images/residual.png}
    \caption{Residual block of ResNet. The weight layers will typically be convolutional layers. Many of these blocks are used throughout ResNet \cite{resnet}.}
    \label{residual}
\end{figure}
ResNet is a CNN from Microsoft research that achieved first place in the Imagenet competition in 2015. The main feature of this model is what is referred to as residual learning. Residual learning involves creating additional connections from one layer of the network to a deeper layer of the network bypassing several of the layers. The reason this is a useful technique is because a common problem with training deep neural networks is gradient explosion, which occurs when large gradients are multiplied by other larger gradients when optimizing a network. During this process, the gradients will become too large to be useful for training the network. Residual connections have the effect of allowing gradients to bypass several layers during training, which prevents the gradients from being multiplied as many times; therefore mitigating gradient explosion \cite{resnet}.

\subsubsection{DenseNet}
\begin{figure}
    \centering
    \includegraphics[width=0.65\textwidth]{images/denseblock.png}
    \caption{A dense block of 5 layers each with a growth rate of $k = 4$ \cite{densenet}.}
    \label{denseblock}
\end{figure}
DenseNet takes the idea of having residual connections from ResNet even further: all layers are directly connected with one another. Further, rather than summing the feature maps like in ResNet, DenseNet concatenates the maps instead. This high connectivity paired with concatenation allows maximum information flow in the network. One would think that the concatenation of the maps of each layer would lead to an excessive number of maps by the end of the network, but DenseNet achieves state-of-the-art results with fairly narrow layers ($k = 32$, where $k$ represents the number of output maps of each convolutional layer). In actuality, the dense connectivity requires fewer parameters than other CNNs because there is no need to re-learn redundant feature maps. One concern of having these connections is that all layers must have the same input and output size. Therefore, in order to allow downsizing of the feature maps, the authors consolidate these connected layers into a block that is referred to as a \emph{dense block}, as seen in Figure \ref{denseblock}. This allows the alternation between dense blocks and down-sampling layers to get high connectivity while allowing downsizing \cite{densenet}.


\subsubsection{Inception}
\begin{figure}
    \centering
    \includegraphics[width=0.50\textwidth]{images/inception.png}
    \caption{Inception block of Inception. It reduces the grid-size of the input map while increasing the number of filters \cite{inception}.}
    \label{inception_block}
\end{figure}
InceptionV3, which we will refer to as Inception from now on, is a CNN model that comes out of Google's research team. It has achieved the highest performance on Imagenet out of all of the models presented in this paper. Inception was designed to perform well under memory and computational constraints with the authors emphasizing a significant boost in performance compared to VGG. The main feature of this architecture is the Inception module. The Inception module is a section of the CNN that consists of concatenation of convolutions of various sizes. This has the effect of a reduction of the number of parameters in the network resulting in a reduction of parameters and, ultimately, a reduction in memory and computation \cite{inception}.


\subsubsection{Single Shot MultiBox Detector}
Single Shot MultiBox Detector (SSD) is a CNN that differs from the previous models in that it is used for detection tasks. This means that the CNN will not only determine the class of the object that appears in the image, but also determine the region in which the object appears. Furthermore, detectors can find multiple instances of a class or even different classes in the same image. SSD works by generating scores for the presence of each class in each default box, where boxes are a predetermined area of the image that an object can appear in. The network will also output adjustments to the box dimensions to create a better fit for the object. Experiments show that SSD is faster and more accurate than other state-of-the-art models on PASCAL VOC, a detection dataset. Impressively, the network is capable of running in real time: 59 frames per second for images of size 300x300 \cite{ssd}.

% Should we add RCNN or YOLO?

\subsubsection{FC-DenseNet}
\begin{figure}
    \centering
    \includegraphics[width=0.40\textwidth]{images/fc-densenet.png}
    \caption{The architecture of FC-DenseNet \cite{fc-densenet}.}
    \label{fc-densenet}
\end{figure}
FC-DenseNet is a CNN that is used for semantic segmentation tasks, which involves determining a label for each pixel in the input image. As seen in Figure \ref{fc-densenet}, the architecture matches that of DenseNet by alternating dense blocks and transition-down blocks. The architecture differs by introducing a mirror of DenseNet where transition-down layers are replaced with transition-up layers, which are transposed convolution layers that upsample the previous feature maps. The most notable contribution of this architecture are the connections between the downsampling and upsampling paths. These connections help the upsampling path recover fine-grained information lost from the transition-down blocks \cite{fc-densenet}.

\subsection{Adam}
Adam is an algorithm for gradient-based optimization of functions. The algorithm draws inspiration from two previously created algorithms: RMSProp and Momentum. By combining these methods, Adam is able to bring a function to convergence more quickly than other optimizers in empirical results \cite{adam}. As a result of its high performance, Adam has become the most recommended optimizer in the deep learning community.
\subsection{Batch Normalization}
Batch Normalization (BN) is an additional layer in a neural network that performs normalization for each training batch. This method allows the use of higher learning rates, which leads to decreased training time. Furthermore, the authors' empirical results show increased accuracy in models using Batch Normalization \cite{batchnorm}.
\subsection{Dropout}
Dropout is a simple method used to prevent neural networks from overfitting. The technique involves randomly omitting nodes in a neural network during training time \cite{dropout}. The probability of node being omitted is controlled by a parameter $p$. Though counter-intuitive, this creates a better model because it forces all of the nodes of the network to contribute to predicting. In other words, the network cannot simply rely on a few strong nodes to make predictions as they may be randomly omitted. % Literature Survey

\chapter{Research Goals}

The goal of this work is to build and train algorithms to analyze aerial imagery of lettuce and orange plants. This imagery will include both hyperspectral and RGB data so that comparisons can be made between the two. In the analysis we will attempt to predict plant health, where water potential and chlorophyll content are key indicators of such. The indicators will be collected using a water potential meter and a chlorophyll meter. 

The research goals are as follows:
\begin{enumerate}
    \item Using remote sensing, create datasets of hyperspectral and RGB imagery of lettuce and orange plants
    \item Collect ground truthing data and associate each value with the plant's ID
    \item Create localization algorithm to crop plant from imagery and produce plant ID
    \item Train convolutional neural network to predict plant health from imagery
    \item Train an algorithm to predict NDVI values from RGB imagery
\end{enumerate} % Research Goals

\chapter{Methodology}

%Algorithm training process:
%\begin{enumerate}
%\item Plants in the imagery will be localized with Faster R-CNN \cite{r-cnn} and cropped
%\item Localized plants will be associated with plant ID
%\item Cropped images will be used as training data, where targets are the plant health indicators
%\end{enumerate}

\section{Dataset Creation}
In this section, we go over the instruments and the process that we used to create the dataset.

\subsection{Instruments}
For the UAV, we utilized two different systems: Aibotix's Aibot X6 and PrecisionHawk's Lancaster 5 as pictured in Figures \ref{aibotx6} and \ref{lancaster5}. The Aibot X6 is a multirotor UAV equipped with a Headwall Nano-Hyperspec sensor. The Lancaster 5 is a fixed-wing UAV equipped with both a multispectral sensor and an RGB sensor.


\begin{figure}
    \centering
    \includegraphics[width=0.8\textwidth]{images/aibotx6.JPG}
    \caption{Aibot X6 from Aibotix collecting data from our lettuce crops.}
    \label{aibotx6}
\end{figure}
\begin{figure}
    \centering
    \includegraphics[width=0.8\textwidth]{images/lancaster5.JPG}
    \caption{Lancaster 5 from PrecisionHawk.}
    \label{lancaster5}
\end{figure}

To validate the multispectral data from the UAV, we used an ASD HandHeld 2 Spectroradiometer. We used the optional leaf clip attachment for more precise measurements.

To collect chlorophyll data, we used the Konica Minolta SPAD 502. It is a handheld device that measures the amount of chlorophyll in leaves, an indicator of how well fertilized a plant is.

The water potential data was determined using the Decagon WP4C. Leaf cutlets are inserted into the device where the readings are determined. The data is an indicator of how well the plant is being watered.



\subsection{Crop Treatment}
The lettuce plants were subjected to 16 different irrigation and nitrogen treatments. The irrigation levels were 0\%, 25\%, 50\% and 100\% of required water according to evapotranspiration. The nitrogen levels were 0\%, 25\%, 50\% and 100\% of required amount for optimal lettuce growth based on soil chemical analysis. This results in the 16 different treatments from the combinations of the 4 levels of irrigation and nitrogen. Each treatment had 3 replications giving us 48 experimental plots as seen in Figure \ref{lettuce_plot}.


\begin{figure}
    \centering
    \includegraphics[width=1.0\textwidth]{images/plot.png}
    \caption{Experimental design of lettuce plot.}
    \label{lettuce_plot}
\end{figure}

\subsection{Data Collection Process}
Data was collected weekly at around noon as having direct sunlight is optimal for the remote sensing data. For the ground truthing data, we sampled 2 plants from each plot giving us data on 96 plants total.
With these 96 plants, we collect data from them using a handheld RGB camera, spectroradiometer, and chlorophyll meter. Since the water potential meter takes more time to process each sample than the other sensors, we were only able to get 1 sample per plot per week, resulting in 48 samples per week. % Plant height, soil moisture, mass

\subsection{UAV Image Processing}
As there are discrepancies in the UAV images due to inaccuracies in the GPS sensor and usage of multiple sensors, processing of the images is necessary. The first step of this process is to perform georeferencing in QGIS. This step will ensure that the coordinates embedded in the GeoTIFF images are properly aligned with the objects in the image. Using a manually generated shapefile, we crop the images so that they all display the same area. With these cropped images, we rotate them so that the rows of lettuce are horizontal. Finally, we resize the images to the dimensions of the smallest image.


\section{Models}

\subsection{Ground Truth Predictor}
With this model, we experiment with the various architectures listed previously: VGG, ResNet, DenseNet, and Inception. We utilize a pre-trained network and replace the final layers with our own.

\subsection{NDVI Predictor}
The goal of this model is to take as input an aerial RGB image and return an image of the same dimensions where each pixel value represents the level of NDVI. For this model, we modify FC-DenseNet so that it predicts a value from $[0, 256)$, which is a binned representation of the NDVI value.

\chapter{Evaluation of Results}


Models will be evaluated with mean absolute percentage error (MAE) or accuracy where appropriate.
\begin{equation} MAE = \frac{1}{n} \sum_{i=1}^{n} |y-\hat{y}| \end{equation}

To determine the practicability of the models, we will also evaluate how long the model takes to analyze images with and without a GPU. All evaluation was conducted on an Intel i5-6600K for the CPU and an NVIDIA GTX 1070 for the GPU.

\chapter{Timeline}
\begin{tabular}{ l l l }
Quarter & Weeks & Task \\
Fall & 10-11 & Data collection and processing \\
Break & 1-3 & Data processing and algorithm implementation \\ 
Winter & 1-4 & Algorithm implementation and training \\
Winter & 5-8 & Complete thesis write-up \\
Winter & 9-10 & Thesis defense
\end{tabular} % Timeline

%\chapter{Timeline}
\begin{tabular}{ l l l }
Quarter & Weeks & Task \\
Fall & 10-11 & Data collection and processing \\
Break & 1-3 & Data processing and algorithm implementation \\ 
Winter & 1-4 & Algorithm implementation and training \\
Winter & 5-8 & Complete thesis write-up \\
Winter & 9-10 & Thesis defense
\end{tabular} % Results and Discussion

%\input{Chapters/Chapter7} % Conclusion

%% ----------------------------------------------------------------
% Now begin the Appendices, including them as separate files

\backmatter

%% ----------------------------------------------------------------
\label{Bibliography}
\lhead{\emph{Bibliography}}  % Change the left side page header to "Bibliography"
\bibliographystyle{unsrtnat}  % Use the "unsrtnat" BibTeX style for formatting the Bibliography
\bibliography{Bibliography}  % The references (bibliography) information are stored in the file named "Bibliography.bib"

\end{document}  % The End
%% ----------------------------------------------------------------